\documentclass[11pt]{article}
\usepackage{enumitem}
\usepackage{amssymb}
\usepackage{amsmath}
\usepackage{color}

\newcommand{\red}[1]{{\color{red}#1}}
\newcommand{\green}[1]{{\color{green}#1}}
\newcommand{\blue}[1]{{\color{blue}#1}}
\newcommand{\SFW}[3]{{\color{blue}}}

\newcommand{\redbullet}{$\red{\bullet}$}
\newcommand{\bluebullet}{$\blue{\bullet}$}

\newtheorem{example}{Example}

\begin{document}

\title{B561 Advanced Database Concepts \\Assignment 2 \\Fall 2021}
\author{Dirk Van Gucht}
\date{}
\maketitle

This {assignment} relies on the lectures
\begin{itemize}
\item SQL Part 1 and SQL Part 2 (Pure SQL);
\item Views;
\item Relational Algebra (RA);  and 
\item Joins and semijoins. 
\end{itemize}

Furthermore, the lecture on the correspondence between Tuple Relational Calculus and SQL should help you to solve problems in this assignment.

To turn in your assignment, you will need to upload to Canvas a single file with name {\tt assignment2.sql} which contains 
the necessary SQL statements that solve the problems in this assignment.   
The {\tt assignment2.sql} file must be so that the AI's can run it in their PostgreSQL environment.  
You should use the {\tt Assignment2Script.sql} file to construct the {\tt assignment2.sql} file. (Note that the data to be used for this assignment is included in this file.)
In addition, you will need to upload a separate {\tt assignment2.txt} file that contains the results of running
your queries.
Finally, you need to upload a file {\tt assignment2.pdf} that contains the solutions to the problems that require it.


The problems that need to be included in the {\tt assignment2.sql} are marked with a blue bullet \bluebullet.
The problems that need to be included in the {\tt assignment2.pdf} are marked with a red bullet \redbullet.
(You should aim to use Latex to construct your .pdf file.)

\newpage
\noindent
\large{\bf Database schema and instances}
\bigskip




For the problems in this assignment we will use the following database schema:\footnote{The primary key, which may consist of one or more attributes, of each of these relations is underlined.}

\begin{center}
{\tt
  \begin{tabular}{l}
  {Person}($\underline{\tt pid}$, pname, city) \\
  {Company}($\underline{\tt cname}, {\tt headquarter}$) \\
  {Skill}($\underline{\tt skill}$) \\
  {worksFor}($\underline{\tt pid}$, cname, salary) \\
  {companyLocation}($\underline{\tt cname, city}$) \\
  {personSkill}($\underline{\tt pid, skill}$) \\
  {hasManager}($\underline{\tt eid, mid}$) \\
  {Knows}($\underline{\tt pid1, pid2}$) \\
   \end{tabular}
  }
\end{center}

In this database we maintain a set of persons ({\tt Person}), a set
of companies ({\tt Company}), and a set of (job) skills ({\tt Skill}).  
The {\tt pname} attribute in {\tt Person} is the name of the person.  
The {\tt city} attribute in {\tt Person} specifies the city in which the person lives.  
The {\tt cname} attribute in {\tt Company} is the name of the company.
The {\tt headquarter} attribute in {\tt Company} is the name of the city wherein the company has its headquarter.
The {\tt skill} attribute in {\tt Skill} is the name of a (job) skill.

A person can work for at most one company. This information is maintained in the {\tt worksFor} relation. (We permit that a person does not work for any company.)
The {\tt salary} attribute in {\tt worksFor} specifies the salary made by the person.

The {\tt city} attribute in {\tt companyLocation} indicates a city in which the company is located.
(Companies may be located in multiple cities.)

A person can have multiple job skills. This information is maintained in the {\tt personSkill} relation.  A job skill can be
the job skill of multiple persons.  (A person may not have any job skills, and a job skill may
have no persons with that skill.)

A pair $(e,m)$ in {\tt hasManager} indicates that person $e$ has  
person $m$ as one of his or her managers.
We permit that an employee has multiple managers and that a manager  may manage
multiple employees.  (It is possible that an employee has no manager
and that an employee is not a manager.)
We further require that 
an employee and his or her managers must work for the
same company.

The relation {\tt Knows} maintains a set of pairs $(p_1,p_2)$ where $p_1$ 
and $p_2$ are pids of persons.   The pair $(p_1,p_2)$ indicates that the person with
pid $p_1$ knows the person with pid $p_2$.
We do not assume that the relation {\tt Knows} is
symmetric: it is possible that $(p_1,p_2)$ is in the relation but that
$(p_2,p_1)$ is not.

The domain for the attributes {\tt pid}, {\tt pid1}, {\tt pid2}, {\tt salary}, {\tt eid}, and {\tt mid} is {\tt integer}.   The domain for all other attributes is {\tt text}.

We assume the following foreign key constraints:
\begin{itemize}
\item {\tt pid} is a foreign key in {\tt worksFor} referencing the primary key {\tt pid} in {\tt Person};
\item {\tt cname} is a foreign key in {\tt worksFor} referencing the primary key {\tt cname} in {\tt Company};
\item {\tt cname} is a foreign key in {\tt companyLocation} referencing the primary key {\tt cname} in {\tt Company};
\item {\tt pid} is a foreign key in {\tt personSkill} referencing the primary key {\tt pid} in {\tt Person};
\item {\tt skill} is a foreign key in {\tt personSkill} referencing the primary key {\tt skill} in {\tt Skill};
\item {\tt eid} is a foreign key in {\tt hasManager} referencing the primary key {\tt pid} in {\tt Person}; 
\item {\tt mid} is a foreign key in {\tt hasManager} referencing the primary key {\tt pid} in {\tt Person};
\item {\tt pid1} is a foreign key in {\tt Knows} referencing the primary key {\tt pid} in {\tt Person}; and
\item {\tt pid2} is a foreign key in {\tt Knows} referencing the primary key {\tt pid} in {\tt Person}
\end{itemize}

The file {\tt Assignment2Script.sql} contains the data supplied for this assignment.
\newpage

\noindent
\large{\bf Pure SQL and RA SQL}

\bigskip
In this assignemt, we distinguish between Pure SQL and RA SQL.
Below we list the {\bf only} features that are allowed in Pure SQL and in RA SQL. 

In particular notice that 
\begin{itemize}
\item {\tt JOIN}, {\tt NATURAL~JOIN}, and {\tt CROSS~JOIN} are {\bf not} allowed in Pure SQL.
\item The predicates  [{\tt NOT}]~{\tt IN}, {\tt SOME}, {\tt ALL}, [{\tt NOT}]~{\tt EXISTS} are {\bf not} allowed in RA SQL.
\end{itemize}

\begin{center}
\begin{tabular}{c}
{\bf The only features allowed in Pure SQL} \\ \hline
\begin{tabular}{l}
{\tt SELECT ... FROM ... WHERE}\\
{\tt WITH ...} \\
{\tt UNION, INTERSECT, EXCEPT} operations \\
{\tt EXISTS} and {\tt NOT EXISTS} predicates \\
{\tt IN} and {\tt NOT IN} predicates \\
{\tt ALL} and {\tt SOME} predicates \\
{\tt VIEW}s that can only use the above RA SQL features
\end{tabular} \\
 \\
{\bf The only features allowed in Pure SQL features} \\ \hline 
\begin{tabular}{l}
{\tt SELECT ... FROM ... WHERE}\\
{\tt WITH ...} \\
{\tt UNION, INTERSECT, EXCEPT} operations\\
{\tt JOIN ... ON ...}, {\tt NATURAL JOIN}, and {\tt CROSS JOIN} operations \\
{\tt VIEW}s that can only use the above RA SQL features\\ 
commas in the {\tt FROM} clause are {\bf not} allowed \\
\end{tabular} \\
\end{tabular}
\end{center}

\newpage

\section{Formulating queries in Pure SQL with and without set predicates}
\label{PureSQL}

An important consideration in formulating queries is to contemplate how they can be written in different, but equivalent, ways. 
In fact, this is an aspect of programming in general and, as can expected, is also true for SQL.   A learning outcome of this course
is to acquire skills for writing queries in different ways.    One of the main motivations for this is to learn 
that different formulations of the same query can 
dramatically impact performance, sometimes by orders of magnitude.    


For the problems in this section, you will need to formulate queries in {Pure SQL} with and without set predicates.
You can use the SQL operators {\tt INTERSECT}, {\tt UNION}, and {\tt EXCEPT}, unless otherwise specified.
You are however allowed and  encouraged to use views including temporary and user-defined views.

To illustrate what you need to do, consider the following example.
\begin{example}\label{PureSQLquery}
Consider the query ``\emph{Find the pid and name of each person who (a)
works for a company located in Bloomington  and (b)
knows a person who lives in Chicago.}''

\begin{description}
\item[(a)]   Formulate this query in Pure SQL by only using the {\tt EXISTS} or {\tt NOT EXISTS} set predicates.
You can not use the set operations {\tt INTERSECT}, {\tt UNION}, and {\tt EXCEPT}.

A possible solution is
{\footnotesize
\begin{verbatim}
select p.pid, p.pname
from   Person p
where  exists (select 1
               from   worksFor w
               where  p.pid = w.pid and
                      exists (select 1
                              from   companyLocation c
                              where  w.cname = c.cname and c.city = 'Bloomington')) and
       exists (select
               from   Knows k
               where  p.pid = k.pid1 and
                      exists (select 1
                              from   Person p2
                              where  k.pid2 = p2.pid and
                                     p2.city = 'Chicago'));
\end{verbatim}
}

\item[(b)]   Formulate this query in Pure SQL by only using the {\tt  IN}, {\tt NOT IN}, {\tt SOME}, or {\tt ALL} set membership predicates.
You can not use the set operations {\tt INTERSECT}, {\tt UNION}, and {\tt EXCEPT}.

A possible solution is
{\footnotesize
\begin{verbatim}
select p.pid, p.pname
from   Person p
where  p.pid in (select w.pid
                 from   worksFor w
                 where  w.cname in (select c.cname
                                    from   companyLocation c
                                    where  c.city = 'Bloomington')) and
       p.pid in (select k.pid1
                 from   Knows k
                 where  k.pid2 in (select p2.pid
                                   from   Person p2
                                   where  p2.city = 'Chicago'));
\end{verbatim}
}

Another possible solution using the {\tt SOME} and {\tt IN} predicates
{\footnotesize
\begin{verbatim}
select p.pid, p.pname
from   Person p
where  p.pid = some (select w.pid
                     from   worksFor w
                     where  w.cname = some (select c.cname
                                            from   companyLocation c
                                            where  c.city = 'Bloomington')) and
       p.pid in (select k.pid1
                 from   Knows k
                 where  k.pid2 in (select p2.pid
                                   from   Person p2
                                   where  p2.city = 'Chicago'));
\end{verbatim}
}

\item[(c)]   Formulate this query in Pure SQL without using set predicates.
A possible solution is
{\footnotesize
\begin{verbatim}
select p.pid, p.pname
from   Person p, worksFor w, companyLocation c
where  p.pid = w.pid and
       w.cname = c.cname and
       c.city = 'Bloomington'
intersect
select p1.pid, p1.pname
from   Person p1, Knows k, Person p2
where  k.pid1 = p1.pid and
       k.pid2 = p2.pid and
       p2.city = 'Chicago';
\end{verbatim}}

\end{description}
\end{example}

\newpage
We now turn to the problems for this section.
\begin{enumerate}[resume]
\item\label{queryTwo}   
Consider the query ``\emph{Find the pid and name of each person who works for Google and who has a strictly higher salary
than some other person who he or she knows and who also works for Google}."

\begin{enumerate}
\item  \bluebullet\ Formulate this query in Pure SQL by only using the {\tt EXISTS} or {\tt NOT EXISTS} set predicates.

\item \bluebullet\ Formulate this query in Pure SQL by only using the {\tt  IN}, {\tt NOT IN}, {\tt SOME}, or {\tt ALL} set membership predicates.

\item  \bluebullet\  Formulate this query in Pure SQL without using set predicates.

\end{enumerate}

\item\label{queryThree}  Consider the query ``\emph{Find the cname of each company with headquarter in Cupertino, but that is not located in Indianapolis, along with the pid, name, and salary
of each person who works for that company and who has the next-to-lowest salary 
at that company.}''\footnote{By definition, a salary $s$ is next-to-lowest if there exists salary $s_1$ with $s_1<s$, but there do not exist
two salaries $s_1$ and $s_2$ with $s_2<s_1<s$.}


\begin{enumerate}
\item  \bluebullet\  Formulate this query in Pure SQL by only using the {\tt EXISTS} or {\tt NOT EXISTS} set predicates.
You can not use the set operations {\tt INTERSECT}, {\tt UNION}, and {\tt EXCEPT}.

\item  \bluebullet\  Formulate this query in Pure SQL by only using the {\tt  IN}, {\tt NOT IN}, {\tt SOME}, or {\tt ALL} set membership predicates.
You can not use the set operations {\tt INTERSECT}, {\tt UNION}, and {\tt EXCEPT}.

\item    \bluebullet\  Formulate this query in Pure SQL without using set predicates.

\end{enumerate}


\item \label{queryFour} 
Consider the query ``\emph{Find each $(c,p)$ pair where $c$ is the cname of a company and $p$ is the pid of a person who works
for that company and who is known by all other persons who work for that company}.
\begin{enumerate}
\item  \bluebullet\   Formulate this query in Pure SQL by only using the {\tt EXISTS} or {\tt NOT EXISTS} set predicates.
You can not use the set operations {\tt INTERSECT}, {\tt UNION}, and {\tt EXCEPT}.

\item  \bluebullet\ Formulate this query in Pure SQL by only using the {\tt  IN}, {\tt NOT IN}, {\tt SOME}, or {\tt ALL} set membership predicates.
You can not use the set operations {\tt INTERSECT}, {\tt UNION}, and {\tt EXCEPT}.

\item  \bluebullet\   Formulate this query in Pure SQL without using  set predicates.

\end{enumerate}

\item \label{queryFive} Consider the query ``\emph{Find each skill that is not a jobskill of any person who works for Yahoo or for Netflix}."
\begin{enumerate}

\item  \bluebullet\   Formulate this query in Pure SQL using  subqueries and set predicates.
You can not use the set operations {\tt INTERSECT}, {\tt UNION}, and {\tt EXCEPT}.

\item   \bluebullet\  Formulate this query in Pure SQL without using predicates.
\end{enumerate}

\item \label{querySeven} Consider the query ``\emph{Find the pid and name of each person who manages all-but-one person who works for Google.}
\begin{enumerate}
\item  \bluebullet\   Formulate this query in Pure SQL using subqueries and set predicates.
You can not use the set operations {\tt INTERSECT}, {\tt UNION}, and {\tt EXCEPT}.

\item   \bluebullet\  Formulate this query in Pure SQL without using set predicates.

\end{enumerate}
\end{enumerate}

\newpage

\section{Formulating queries in Relational Algebra and RA SQL}

Reconsider the queries in Section~\ref{PureSQL}.   The goal of the problems in this section
is to formulate these queries in Relational Algebra in standard notation and in RA SQL.  

There are some further restrictions:
\begin{itemize}
\item You can only use {\tt WHERE} clauses that use 
conditions involving constants.   For example conditions of the form ``$t.A\, \theta\, \text{'a'}$'' are allowed, but conditions of the 
form '$t.A\, \theta\, s.B$' are not allowed.   The latter conditions can be absorbed in {\tt JOIN} operations in the {\tt FROM} clause.
\item You can not use commas in any {\tt FROM} clause.  Rather, you should use {\tt JOIN} operations.
\end{itemize}

You can use the following letters, or indexed letters, to denote relation names in RA expressions:

\begin{center}
\fbox{
\begin{tabular}{l|l}
$P$, $P_1$, $P_2, \cdots$ & {\tt Person} \\
$C$, $C_1$, $C_2, \cdots$ & {\tt Company} \\
$S$, $S_1$, $S_2, \cdots$ & {\tt Skill} \\
$W$, $W_1$, $W_2, \cdots$ & {\tt worksFor} \\
$cL$, $cL_1$, $cL_2, \cdots$ & {\tt companyLocation} \\
$pS$, $pS_1$, $pS_2, \cdots$ & {\tt personSkill} \\
$M$, $M_1$, $M_2, \cdots$ & {\tt hasManager} \\
$K$, $K_1$, $K_2, \cdots$ & {\tt Knows} \\
\end{tabular}
}
\end{center}

To illustrate what you need to do reconsider the query in Example~\ref{PureSQLquery} in Section~\ref{PureSQL}.

\begin{example}\label{RAquery}
Consider the query ``\emph{Find the pid and name of each person who (a)
works for a company located in Bloomington  and (b)
knows a person who lives in Chicago}.''

\begin{description}
\item[(a)]  Formulate this query in Relational Algebra in standard notation.

A possible solution is
{\tiny
\[
\begin{array}{l}
\pi_{pid,pname}(Person\bowtie worksFor \bowtie\, \pi_{cname}(\sigma_{city = \mathbf{Bloomington}}(companyLocation))) \cap\\
\qquad 
\pi_{Person_1.pid,Person_1.pname}(Person_1 \bowtie_{Person_1.pid = pid1} Knows \bowtie_{pid2 = Person_2.pid}\,\pi_{Person_2.pid}(\sigma_{city = \mathbf{Chicago}}(Person_2)))
\end{array}
\]
}

If we use the letters in the above box, this expression becomes more succinct:

{\tiny
\[
\begin{array}{l}
\pi_{pid,pname}(P\bowtie W \bowtie\, \pi_{cname}(\sigma_{city = \mathbf{Bloomington}}(cL))) \cap\\
\qquad 
\pi_{P_1.pid,P_1.pname}(P_1 \bowtie_{P_1.pid = pid1} K \bowtie_{pid2 = P_2.pid}\,\pi_{P_2.pid}(\sigma_{city = \mathbf{Chicago}}(P_2)))
\end{array}
\]
}

You are also allowed to introduce letters that denote expressions.  For example, let $E$ and $F$ denote the expression
\[ \pi_{pid,pname}(P\bowtie W \bowtie\, \pi_{cname}(\sigma_{city = \mathbf{Bloomington}}(cL)))\]
and
\[\pi_{P_1.pid,P_1.pname}(P_1 \bowtie_{P_1.pid = pid1} K \bowtie_{pid2 = P_2.pid}\,\pi_{P_2.pid}(\sigma_{city = \mathbf{Chicago}}(P_2))),\]
respectively. Then we can write the solution as follows:
\[E\cap F.\]

\item[(b)]  Formulate this query in RA SQL.   

A possible solution is
{\footnotesize
\begin{verbatim}
select pid, pname
from   Person
       NATURAL JOIN worksFor
       NATURAL JOIN (select cname 
                     from   companyLocation
                     where  city = 'Bloomington') C
INTERSECT
select P1.pid, P1.pname
from   Person P1 
       JOIN Knows ON (P1.pid = pid1)
       JOIN (select pid
             from   Person
             where  city = 'Chicago') P2 ON (pid2 = P2.pid)
order by 1,2;
\end{verbatim}
}                    

Observe that the {\tt WHERE} clauses only use conditions involving constants.

\end{description}
\end{example}

\newpage

We now turn to the problems in this section.
\begin{enumerate}[resume]
\item Reconsider Problem~\ref{queryTwo}.
Find the pid and name of each person who works for Google and who has a strictly higher salary
than some other person who he or she knows and who also works for Google.
\begin{enumerate}
\item  \redbullet\  Formulate this query in Relational Algebra in standard notation.

\item \bluebullet\  Formulate this query in RA SQL.
\end{enumerate}

\item Reconsider Problem~\ref{queryThree}.
Find the cname of each company with headquarter in Bloomington, but not located in Indianapolis, along with the pid, name, and salary
of each person who works for that company and who has the next-to-lowest salary
at that company.

\begin{enumerate}
\item  \redbullet\ Formulate this query in Relational Algebra in standard notation.


\item   \bluebullet\ Formulate this query in RA SQL.

\end{enumerate}
\item Reconsider Problem~\ref{queryFour}.  Find each $(c,p)$ pair where $c$ is the cname of a company and $p$ is the pid of a person who works
for that company and who is known by all other persons who work for that company.
\begin{enumerate}
\item  \redbullet\ Formulate this query in Relational Algebra in standard notation.


\item   \bluebullet\ 
Formulate this query in RA SQL.
\end{enumerate}

\item Reconsider Problem~\ref{queryFive}. Find each skill that is not a jobskill of any person who works for Yahoo or for Netflix.

\begin{enumerate}
\item  \redbullet\ Formulate this query in Relational Algebra in standard notation.

\item   \bluebullet\ Formulate this query in RA SQL.

\end{enumerate}

\item Reconsider Problem~\ref{querySeven}. Find the pid and name of each person who manages all-but-one person who works for Google.
\begin{enumerate}
\item  \redbullet\ Formulate this query in Relational Algebra in standard notation.


\item  \bluebullet\  Formulate this query in RA SQL.

\end{enumerate}
\end{enumerate}

\newpage
\section{Formulating constraints using Relational Algebra}

Recall that it is possible to express constraints in TRC and as boolean SQL queries.
It is also possible to write constraints using RA expressions.
Let $E$, $F$, and $G$  denote RA expressions. Then we can write RA expression comparisons that express constraints:
\begin{center}
\begin{tabular}{llll}
$E \not=\emptyset$ & which is true if $E$ evaluates to an \blue{non-empty} relation \\
$E = \emptyset$ & which is true if $E$ evaluates to the \blue{empty} relation \\
$F \subseteq G$ & which is true if $F$ evaluates to a relation \\
&that is a \blue{subset} of the relation obtained from $G$\\ 
$F \not\subseteq G$ & which is true if $F$ evaluates to a relation \\
&that is \blue{not} a \blue{subset} of the relation obtained from $G$\\ 
\end{tabular}
\end{center}

Here are some examples of writing constraints in this manner.

\begin{example}\label{constraintOne}
``\emph{Some person works for Google}.''  This constraint can be written as follows:
\[\pi_{pid}(\sigma_{cname = \mathbf{Google}}(worksFor)) \neq \emptyset.
\]
Indeed,  the RA expression \[\pi_{pid}(\sigma_{cname = \mathbf{Google}}(worksFor))\]
computes the set of all person pids who work for Google.   If this set is not empty then there
are indeed persons who work for Google.

Incidentally, the constraint ``\emph{No one works for Google}'' can be written as follows:
\[\pi_{pid}(\sigma_{cname = \mathbf{Google}}(worksFor)) = \emptyset.
\]
\end{example}

\begin{example}\label{constraintTwo}
\emph{Each person has at least two skills}.   This constraint can be written as follows:
\[\pi_{pid}(P) \subseteq \pi_{pS_1.pid}(\sigma_{pS_1.skill \neq pS_2.skill}(pS_1\bowtie_{pS_1.pid = pS_2.pid} pS_2)).
\]
Indeed, \[\pi_{pid}(P)\] computes the set of all person pids and
\[\pi_{pS_1.pid}(\sigma_{pS_1.skill \neq pS_2.skill}(pS_1\bowtie_{pS_1.pid = pS_2.pid} pS_2))\] computes the set of
all pids of persons who have at least two skills.    When the first set is contained in the second we must have that each person has at least two skills.

Incidentally, the constraint ``\emph{Some person has fewer than two skills}" can be written as follows:
\[\pi_{pid}(P) \not\subseteq \pi_{pS_1.pid}(\sigma_{pS_1.skill \neq pS_2.skill}(pS_1\bowtie_{pS_1.pid = pS_2.pid} pS_2)).
\]

\end{example}

We now turn to the problems in this section.

Formulate each of the following constraints using RA expressions as illustrated in Example~\ref{constraintOne} and
Example~\ref{constraintTwo}.

\begin{enumerate}[resume]


\item \redbullet\  Some person has a salary that is strictly lower than that of each of the persons who he or she manages.

\item \redbullet\ No person knows all persons who work for Google.

\item \redbullet\  Each person knows all of his or her managers.

\item \redbullet\  Each employee and his or her managers work for the same company.

\item \redbullet\  The attribute pid is a primary key of the Person relation.

\end{enumerate}




\newpage

\section{Formulating queries in SQL using views}


Formulate the following views and queries in SQL. You are allowed to combine the features of both Pure SQL and RA SQL.

\begin{enumerate}[resume]

\item  \label{viewtriangle} \bluebullet\ 
Create a view {\tt Triangle} that contains each triple of pids of different persons $(p_1,p_2,p_3)$
that mutually know each other.   

Then test your view.
           
\item\label{viewone} \bluebullet\  Define a parameterized view {\tt SalaryBelow(cname text, salary integer)} that returns, for a given
company identified by {\tt cname} and a given {\tt salary} value, the subrelation of
{\tt Person} of persons who work for company {\tt cname} and whose salary is strictly below {\tt salary}.

Test your view for the parameter values ({ 'IBM'},60000), ({'IBM'},50000), and ({'Apple'},65000).

\item \bluebullet\ Define a parameterized view {\tt KnowsPersonAtCompany(p integer, c text)} that returns for a person with pid $p$ 
the subrelation of {\tt Person} of persons who $p$ knows and who work for the company with cname $c$.

Test your view for the parameters (1001, ‘Amazon’), (1001,‘Apple’), and (1015,‘Netflix’).

\item \bluebullet\ Define a parameterized view {\tt KnownByPersonAtCompany(p integer, c text)} that returns the subrelation of {\tt Person} of persons
who know the person with pid $p$ and who work for the company with cname $c$.

Test your view for the parameters (1001, ‘Amazon’), (1001,‘Apple’), and (1015,‘Netflix’).

\item \bluebullet\  Let $Tree(parent: integer, child: integer)$ be a rooted parent-child tree.   So a pair $(n,m)$ in $Tree$ indicates that node $n$ is a parent of node $m$.
The {\tt sameGeneration(n1, n2)} binary relation is inductively defined using the following two rules:
\begin{itemize}
\item If $n$ is a node in $T$, then the pair $(n,n)$ is in the {\tt sameGeneration} relation. ({\bf Base rule})
\item If $n_1$ is the parent of $m_1$ in $Tree$ and $n_2$ is the parent of $m_2$ in $Tree$ and 
$(n_1,n_2)$ is a pair in the {\tt sameGeneration} relation then $(m_1,m_2)$ is a pair in the {\tt sameGeneration} relation. ({\bf Inductive Rule})
\end{itemize}

Write a \blue{recursive view} for the {\tt sameGeneration} relation.

Test your view.
\end{enumerate}


\end{document}





